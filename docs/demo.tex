% -*- coding:utf-8 -*-
% LATEX PREAMBLE --- needs to be imported manually
\documentclass[12pt]{article}
\special{papersize=3in,5in}
\usepackage[utf8]{inputenc}
\usepackage{amssymb,amsmath}
\pagestyle{empty}
\setlength{\parindent}{0in}
\newcommand{\detail}[1]{{\scriptsize(#1)\par}~}
\newcommand{\refs}[1]{{\scriptsize\textit{Refs: }#1\par}\hfill.}
\newcommand*{\abs}[1]{\left\vert#1\right\vert}
\newcommand*{\NN}{\mathbb{N}}
\newcommand*{\ZZ}{\mathbb{Z}}
\newcommand*{\QQ}{\mathbb{Q}}
\newcommand*{\FF}{\mathbb{F}}
\newcommand*{\PP}{\mathbb{P}}
\newcommand*{\units}{\times}

%%% commands that do not need to imported into Anki:
\usepackage{mdframed}
\newcommand*{\tags}[1]{\paragraph{tags: }#1\bigskip}
\newcommand*{\xfield}[1]{\begin{mdframed}\centering #1\end{mdframed}\bigskip}
\newenvironment{field}{}{}
\newcommand*{\xplain}[1]{\begin{mdframed}\texttt{#1}\end{mdframed}\bigskip}
\newenvironment{plain}{\ttfamily}{\par}
\newenvironment{note}{}{}
% END OF THE PREAMBLE
\begin{document}
\tags{number-theory}
\begin{note}
  \xfield{def: Euler \(\phi\)-function}
  \begin{field}
    \[
      \phi(n) = \abs{(\ZZ/n)^\units}
    \]
  \end{field}
\end{note}

\begin{note}
  \xfield{express \(n\in\NN\) in terms of Euler's \(\phi\)-function}
  \begin{field}
    \[n=\sum_{d|n} \phi(d)\]
    
    \detail{Each element of \(\ZZ/n\) is a generator of one of the subgroups \(\ZZ/d\) with \(d|n\).}

    \refs{Serre: A course in arithmetic, \S~I.1, Lemma~1}
  \end{field}
\end{note}

\begin{note}
  \xfield{Chevalley-Warning-Theorem}
  \begin{field}
    Let \(\FF_q\) be a finite field (\(q\) a power of \(p\)).

    The common vanishing locus of polynomials \(f_{\alpha} \in \FF_q[x_1,\dots,x_n]\) of sufficiently small degree has cardinality a multiple of \(p\).  Sufficiently small means that \(\sum_\alpha \deg(f_\alpha) < n\).\\

    \refs{Serre: A course in arithmetic, \S~I.2, Theorem~3}
  \end{field}
\end{note}

\begin{note}
  \xfield{prove: Every conic over a finite field has a rational point.}
  \begin{field}
    A conic in \(\PP^n_{\FF_q}\) is defined by a quadratic form \(f\) in \(n\) variables (\(n\geq 3\)). 
    As \(\deg(f) = 2 < n\), the Chevalley-Warning-Theorem implies that \(f\) has more zeroes than just the trivial zero.\\

    \refs{Serre: A course in arithmetic, \S~I.2, Corollary~2}
  \end{field}
\end{note}

\begin{note}
  \xfield{squares in \(\FF_q\)  (\(q\) a power of \(p\))}
  \begin{field}
    If \(p\neq 2\), 
    \begin{alignat*}{5}
      1  \to  (\FF_q^\units)^2 \to 
      &\FF_q^\units & \to & \{\pm 1\} \to 1 \\
      & x          & \mapsto & x^{\frac{q-1}{2}}
    \end{alignat*}
    If \(p=2\), all elements of \(\FF_q\) are squares.\\

    \refs{Serre: A course in arithmetic, \S~I.3, Theorem~4}
  \end{field}
\end{note}

\tags{philosophy}
\begin{note}
  \xplain{What good is it to argue about whether HOLISM or REDUCTIONISM is right?}
  \begin{plain}
    The proper way to understand the matter is to transcend the question, by answering MU.
  \end{plain}
\end{note}
\end{document}